\section{Conclusion}
\label{sec:conclusion}

This paper has presented a detailed mathematical model for an agricultural microgrid suitable for optimization of the operation of the pumps and water usage. The model was implemented in open-source software and results have been presented.

The model is not intended to stand alone but serves as one component of a microgrid energy management system (EMS).
One possible application of this model is for use in model-predictive control (MPC) of the microgrid.
In such a scheme, the system operation is optimized over some time horizon, but only the first step is used as the control output.
The time horizon is then slid forward one step and the optimization is repeated.

In this formulation, a single value represents all the parameters at each time step, including available PV power, loads, desired water volume, all of which are dependent on random processes.
One possible extension of this model would be to make it stochastic so that multiple possible future scenarios are represented.
Unlike in standard model-predictive control (MPC), where the output of the optimization problem is the control input (feed-forward), in stochastic model predictive control (SMPC), the output of the optimization problem is the optimal control law to be applied in the future time periods\cite{mesbah2016}. Thus, the system is able to respond to stochastic events dynamically without having to re-run the MPC problem to obtain new control outputs.
A microgrid control system incorporating SMPC has been presented in \cite{cominesi2015}.