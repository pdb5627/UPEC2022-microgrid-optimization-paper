% Make sure any figures start AFTER the introduction and don't float above it.
\FloatBarrier

\section{Demonstration System}
\label{sec:demo-system}

This paper presents an original mathematical formulation of the operational optimization problem for model-predictive control of an agricultural microgrid. The basic problem is to formulate an optimization problem such that the microgrid can be controlled using a model-predictive controller in the EMS. The optimization problem is specifically formulated for a demonstration system to be installed at the Güneşköy farm\cite{Guneskoy}. A block diagram of the demonstration system is shown in \autoref{fig:demo-system}.

The existing pump will continue to be fed from the grid. The site electrical load will be fed from a hybrid inverter\cite{INVT_manual} that will have battery, PV, and grid connections available. A new pump load will be fed from an inverter drive\cite{Growatt_manual} connected to the PV DC bus.

\begin{figure}
	\centering
	\fontsize{8pt}{9.5pt}\selectfont
	\def\svgwidth{\columnwidth}
	\input{figs/demo_system.pdf_tex}
	\caption{Güneşköy Demonstration System}
	\label{fig:demo-system}
\end{figure}

The power flows within the demonstration system are shown in \autoref{fig:power-flows}.

\begin{figure}
	\centering
	\input{figs/power_flows.pdf_tex}
	\caption{Microgrid power flows}
	\label{fig:power-flows}
\end{figure}





