\section{Implementation}
\label{sec:implementation}

The problem was modeled in the Python programming language using the Pyomo library\cite{hart2011pyomo,bynum2021pyomo} and solved using
%the COIN-OR CBC\cite{CBC}
the SCIP\cite{Scip80}
open-source mixed-integer linear program (MILP) solver. In order to use a MILP solver, auxiliary binary variables were introduced to linearize non-linear functions such as $\max$, $\min$, absolute value, $\ge$, $\le$, and logic functions $\logicand$ and $\logicor$. The formulation of the non-linear functions was based on \cite{YALPMIP_logic} and original work by the author.
%Details are provided in \cref{sec:linearization}.
Semi-continuous variables were modeled using the approach described in \cite{MILP_handout}.

The problem, solved for a time period of \SI{72}{h} with an interval $\Delta t$ of \SI{1}{h}, contained a total of 938 continuous and 1008 binary variables. SCIP converges to a solution in less than 40 seconds.

Initialization to a feasible point was performed using a single forward-pass heuristic of greedy pumping with Pump 2 if PV was available, then meeting loads with PV and BSS energy, discharging BSS to meet loads if necessarily and energy was available. Initial Pump 1 pumping and water use was set by iteratively solving a linear problem to try to add pumping one hour at a time until the desired water usage was met. The linear problem was implemented in Pyomo and solved using GLPK\cite{GLPK}.