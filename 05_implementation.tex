\section{Implementation}

The problem was modeled in the Python programming language using the Pyomo library\cite{hart2011pyomo,bynum2021pyomo} and solved using the COIN-OR CBC\cite{CBC} open-source linear mixed-integer program (MIP) solver. In order to use a MIP solver, auxiliary binary variables were introduced to handle non-linear functions such as $\max$, $\min$, absolute value, $\ge$, $\le$, and logic functions $\logicand$ and $\logicor$. The formulation of the non-linear functions was based on \cite{YALPMIP_logic} and original work by the author. Semi-continuous variables were modeled using the approach described in \cite{MILP_handout}.

% Problem size from logs saved with run 20220317_1333.
The problem, solved for a time period of \SI{72}{h} with an interval $\Delta t$ of \SI{1}{h}, contained a total of 566 continuous and 1008 binary variables. CBC converges near to a solution in less than 30 seconds.

Initialization to a feasible point was performed using a single forward-pass heuristic of greedy pumping with Pump 2 if PV was available, then meeting loads with PV and BSS energy, discharging BSS to meet loads if necessarily and energy was available. Initial Pump 1 pumping and water use was set by iteratively solving a linear problem to try to add pumping one hour at a time until the desired water usage was met. The linear problem was implemented in Pyomo and solved using GLPK\cite{GLPK}.